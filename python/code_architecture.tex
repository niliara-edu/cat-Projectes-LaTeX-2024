\section{Arquitectura del codi}
El codi a Python ha estat fet amb l'objectiu de ser f\`acil de llegir,
separat en diversos arxius i utilitzant noms significatius.
Les l\'inies de codi individuals no haurien de
donar cap problema al intentar entendre-les.


\subsection*{Estructura d'un programa de pyxel}
La llibreria Pyxel fa molt d'\'us d'estructures de classes.
Aquestes se separen en tres funcions, per convenci\'o
(i, en el cas de la classe principal, perque la llibreria ho requereix):

\begin{itemize}
\item{\_\_init\_\_:
	S'executa al instanciar la classe, es pot tornar a executar manualment.
	En aquesta funci\'o s'inicialitzen totes les variables inicials amb el seu valor corresponent
	(en el cas de la classe principal, tamb\'e s'inicia pyxel).
	}
\item{update:
	S'executa cada torn. 
	Per convenci\'o, aqu\'i s'executa tot el que t\'e a veure amb
	el funcionament intern del programa i acualitzar les variables.
	}
\item{draw:
	S'executa cada torn.
	Per convenci\'o, aqu\'i s'executa tot el que t\'e a veure amb
	gr\'afics de Pyxel.
}
\end{itemize}


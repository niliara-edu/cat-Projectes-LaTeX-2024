\section{Arquitectura del codi}

\subsection{Estructura principal}
La llibreria Pyxel fa molt d'\'us d'estructures de classes.
Aquestes se separen en tres funcions:

\begin{itemize}
\item{ready:
	S'executa al instanciar la classe, es pot tornar a executar manualment.
	En aquesta funci\'o s'inicialitzen totes les variables inicials amb el seu valor
	corresponent, s'inicien classes, etc.
	}
\item{update:
	S'executa cada torn. 
	Per convenci\'o, aqu\'i s'executa tot el que t\'e a veure amb
	el funcionament intern del programa i acualitzar les variables.
	}
\item{draw:
	S'executa cada torn.
	Per convenci\'o, aqu\'i s'executa tot el que t\'e a veure amb
	gr\'afics de Pyxel.
}
\end{itemize}

Per convenci\'o, hauria de ser \_\_init\_\_ en comptes de ready,
per\`o ho hem substituit per ready per
tenir m\'es control de quan s'executa i per
utilitzar el mateix format que la majoria de game engines
(com Pico-8, la consola de la qual Pyxel s'ha copiat).

En el cas del nostre joc,
utilitzem una classe principal, les quals funcions s'executen per la llibreria Pyxel,
que s'encarrega d'executar les funcions dels nodes fills (escenaris),
els quals s'encarreguen d'executar les funcions dels seus nodes fills.


\subsection{Escenaris}

Els escenaris serveixen per separar apartats del joc
ja que, per exemple, el men\'u no t\'e gaireb\'e res a veure amb el joc.
Com que t\'enen comportaments tan diferents, s'han separat en escenaris.


\subsubsection{Llistes d'entitats}
En l'escenari del joc, les entitats del jugador es guarden com a variable individual,
ja que nom\'es t\'enen una inst\`ancia.
Les entitats que t\'enen m\'es inst\`ancies s'emmagatzemen en llistes,
les quals s\'on iterades posteriorment per
actualitzar cada una de les entitats guardades.

Hi ha dues llistes d'entitats:
la d'enemics actius i la d'enemics inactius (morts).
Quan un enemic mor, aquest es mou a la llista d'enemics morts
fins que arribi el seu moment de reapar\`eixer.
Cada llista executa funcions diferents de les entitats que cont\'e.

\subsubsection{Co\lgem isions}
Per co\lgem isionar les entitats enemigues amb la classe del jugador o l'espasa,
s'itera cada enemic actiu i es comprova la dist\`ancia entre aquests.
En cas de co\lgem isi\'o amb la classe espasa,
l'entitat enemiga va a la llista d'enemics morts,
per\`o no sense abans haver comprobat la co\lgem isi\'o amb el jugador,
que treuria vida.

\subsubsection{Sistema d'events}
Per administrar variables relacionades amb temps i contadors,
s'utilitzen les classes rellotge
(per calcular temps i events d'aparicions)
i contador
(per calcular la quantitat d'enemics actius, restants, i variables de la ronda).


\subsection{Entitats}
Les entitats son classes amb variables i funcions.
Segueixen la convenci\'o de pyxel en el funcionament de les funcions principals,
pel que consisteixen en tres funcions principals:
ready, update i draw.

Les entitats consisteixen b\`asicament en actualitzen la seva posici\'o a update
i printen el seu sprite a l'escenari amb draw.

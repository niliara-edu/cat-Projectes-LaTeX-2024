\section{Arquitectura del codi}

\subsection{Estructura principal}
La llibreria Pyxel fa molt d'\'us d'estructures de classes.
Aquestes se separen en tres funcions, per convenci\'o
(i, en el cas de la classe principal, perque la llibreria ho requereix):

\begin{itemize}
\item{\_\_init\_\_:
	S'executa al instanciar la classe, es pot tornar a executar manualment.
	En aquesta funci\'o s'inicialitzen totes les variables inicials amb el seu valor
	corresponent (en el cas de la classe principal, tamb\'e s'inicia pyxel).
	}
\item{update:
	S'executa cada torn. 
	Per convenci\'o, aqu\'i s'executa tot el que t\'e a veure amb
	el funcionament intern del programa i acualitzar les variables.
	}
\item{draw:
	S'executa cada torn.
	Per convenci\'o, aqu\'i s'executa tot el que t\'e a veure amb
	gr\'afics de Pyxel.
}
\end{itemize}

En el cas del nostre joc,
utilitzem una classe inicial que s'encarrega d'executar
les funcions update() i draw() de la classe de
l'entorn que volem executar en cada cas (section).

Els entorns s'encarreguen tamb\'e d'executar les funcions update() i draw()
de cada classe interna que ho necessiti.
Aquests estan separats perque fan funcions completament diferents,
ja que, per exemple, el men\'u \'es completament independent de la secci\'o del joc.

La majoria d'entorns t\'enen un funcionament simple i intuitiu,
a excepci\'o de l'entorn del joc.


\subsection{Entorn: game.py}

Aquest \'es l'entorn principal del joc.
S'encarrega d'actualitzar events, gr\'afics i entitats del joc.


\subsubsection{Llistes d'entitats}
Les entitats del jugador es guarden com a variable individual,
ja que nom\'es t\'enen una inst\`ancia.
Les entitats que t\'enen m\'es inst\`ancies s'emmagatzemen en llistes,
les quals s\'on iterades posteriorment per
actualitzar cada una de les entitats guardades.

Hi ha dues llistes d'entitats:
la d'enemics actius i la d'enemics inactius (morts).
Quan un enemic mor, aquest es mou a la llista d'enemics morts
fins que arribi el seu moment de reapar\`eixer.
Cada llista executa funcions diferents de les entitats que cont\'e.

\subsubsection{Co\lgem isions}
Per co\lgem isionar les entitats enemigues amb la classe del jugador o l'espasa,
s'itera cada enemic actiu i es comprova la dist\`ancia entre aquests.
En cas de co\lgem isi\'o amb la classe espasa,
l'entitat enemiga va a la llista d'enemics morts,
per\`o no sense abans haver comprobat la co\lgem isi\'o amb el jugador,
que treuria vida.

\subsubsection{Sistema d'events}
Per administrar variables relacionades amb temps i contadors,
s'utilitzen les classes rellotge
(per calcular temps i events d'aparicions)
i contador
(per calcular la quantitat d'enemics actius, restants, i variables de la ronda).



\subsection{Entitats}
Les entitats son classes amb variables i funcions.
Segueixen la convenci\'o de pyxel en el funcionament de les funcions principals,
pel que consisteixen en tres funcions principals:
\_\_init\_\_, update i draw.

Les seves variables m\'es importants s\'on els vectors de posici\'o,
ja que s\'on necessaris per saber a on imprimir la entitat
i comprovar co\lgem isions.
La majoria de la resta del funcionament \'es ll\'ogic i dep\`en \'unicament
de la classe en si.

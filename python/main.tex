\documentclass{article}

\usepackage[catalan]{babel}
\usepackage[utf8]{inputenc}
\usepackage[T1]{fontenc}
\usepackage{hyperref}
\hypersetup{
    colorlinks=true,
    linkcolor=blue,
    filecolor=magenta,      
    urlcolor=cyan
}



\title{Projecte de Big Data - Creaci\'o d'un joc amb Pyxel}
\author{Nil Jimeno, Bernat Brucet}

\begin{document}
\maketitle

\subsection*{todo:}
\begin{itemize}
	\item{Justificación del proyecto: explicación de las necesidades (personales y/o
		sociales) a las que viene a responder el proyecto.}
	\item{Explicación de la arquitectura de la aplicación desarrollada.}
	\item{Referencia a los materiales de terceros utilizados: origen de los fragmentos de
		código ajenos, del material multimedia incluido, de la información técnica
		consultada, etc.}
	\item{Lista de propuestas de mejora de la aplicación: corrección de errores (fixes),
		posibles nuevas funcionalidades, escalabilidad de la aplicación...}
\end{itemize}

\subsection*{Sobre el document:}

Aquest pdf ha estat creat en \LaTeX.

\section{Introducci\'o}
El nostre grup ha fet un joc simple amb Python utilitzant Pyxel.
Aquesta llibreria se centra molt en la ll\`ogica del llenguatge de programaci\'o
i interfereix m\'inimament en el desenvolupament,
excepte a l'hora de mostrar gr\'afics per pantalla.
D'aquesta manera, el projecte no est\`a tant centrat en l'\'us d'abstracci\'o
i memoritzar les funcions de la llibreria,
sino en entendre i posar en pr\`actica la ll\`ogica del programa.
Igualment, es pot permetre funcionar en un llenguatge com Python
ja que el backend est\`a en Rust,
un llenguatge de baix nivell amb un rendiment decent.



\section{Justificaci\'o del projecte}
El nostre projecte s'ha fet amb la intenci\'o de
posar en pr\`actica coneixements relacionats amb Python,
per entendre millor la ll\`ogica del llenguatge de programaci\'o
i aprendre i posar en pr\`actica diferents llibreries.
A part de l'inter\`es personal,
est\`a centrat principalment en aprendre tant
Python com a llenguatge de programaci\'o com l'\'us i interacci\'o de llibreries,
no t\'e cap objectiu m\'es enll\`a del coneixement obtingut.


\subsection{Ll\'ogica abans d'abstracci\'o}
El projecte que hem fet est\`a centrat en la llibreria Pyxel.
Un dels punts que la distingeix per sobre d'altres llibreries, com Pygame,
\'es la poca abstracci\'o que utilitza.
D'aquesta manera, el programa est\`a molt m\'es centrat en la ll\`ogica
del programa i l'\'us d'estructures apropiades per posar-lo en pr\`actica.

Encara que tamb\'e hem fet \'us d'altres llibreries,
la base del projecte es basa molt m\'es en el llenguatge en si que
no pas en funcions de llibreries espec\'ifiques,
aix\'i els coneixements obtinguts no son tan situacionals.


\subsection{Llibreries}
Encara que pyxel no se centra en l'\'us de llibreries externes,
\'es important aprendre a utilitzar-les
perque python est\`a molt centrat en el seu \'us.

\subsubsection{Pyxel}
\'Es la llibreria principal.
S'encarrega de controlar que el programa s'executi en bucle de manera controlada,
crear la finestra de l'aplicaci\'o i mostrar gr\`afics per pantalla.

Encara que sigui per python, la llibreria t\'e un rendiment acceptable
perque ha estat programada en Rust.

\subsubsection{Sqlite3}
\'Es un software d'administrador de bases de dades relacionals a nivell local,
per la qual cosa s'utilitza soving en aplicacions. El programa est\`a escrit en C.
S'encarrega d'administrar algunes dades.

\subsubsection{Flask}
Framework simple fet en python, per projectes poc professionals/tests.
S'encarrega de mostrar una p\`agina amb les puntuacions m\`aximes.


% car\`acters en \k{C}\'atal\`a


\end{document}

\section{Explicaci\'o del codi}

\subsection{HomeController}
El fitxer HomeController.cs \'es l'encarregat de manegar la p\`agina.
Dep\`en de la funci\'o executada, carrega una vista o una altra.


\subsection{Trucades a la api}
Per aconseguir els valors per mostrar a la p\`agina,
primer es fa una trucada a l'api des de HomeController.cs
(aquesta dep\`en de la p\`agina que est\`a carregant i l'argument que ha rebut).

Els resultats de la trucada van a la variable model,
que \'es una inst\`ancia d'una classe amb els par\`ametres que es volen mostrar
i s'utilitza com a argument al carregar la vista.
En el nostre cas, les classes s\'on llistes amb informaci\'o sobre
les begudes que es volen mostrar.


\subsection{Vistes}
Les vistes s\'on l'html rebut per part del controlador,
que \'es el que es mostra a la pantalla del client.
Aquests es fan a partir dels fitxers cshtml,
que ajunten html amb funcions de C\# i ASP.

Algunes d'aquestes funcions utilitzades s\'on el foreach,
que repeteix un codi html per cada beguda que es vulgui mostrar,
o l'if, per comprovar condicions.
Tamb\'e s'utilitza la crida de variables per canviar par\`ametres com el t\'itol i descripci\'o.

Els fitxers de la carpeta home utilitzen comparteixen el codi del layout,
que s'afegeix autom\`aticament a totes les vistes.
Aquest cont\'e configuracions del head, el t\'itol, el navbar i la barra de busca.
\'Es un fitxer compartit per no haver de repetir aquella porci\'o de codi a cada cshtml.


\subsection{Links}
Per anar de p\`agina en p\`agina,
s'utilitzen links amb par\`ametres especials (de cshtml)
que truquen a altres funcions de HomeController.
Per cada funci\'o, es carrega la vista amb aquest mateix nom.
En cas de que requereixin m\'es informaci\'o,
com ara els valors que es volen buscar (com a argument),
s'utilitza asp-target per canviar-los.

D'una manera similar funciona el formulari,
al qual se l'ha d'especificar a quina funci\'o est\`a trucant,
i aquest retornar\'a els valors del formulari en forma d'argument.
Els formularis s\'on utilitzats per fer funcionar la barra de busca.
